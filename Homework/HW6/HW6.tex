%!TEX program = xelatex

\documentclass[12pt,a4paper]{article}
\usepackage{xeCJK}
\usepackage{amsmath}
\setmainfont{Times New Roman}
\usepackage{setspace}
\usepackage{caption}
\usepackage{graphicx, subfig}
\usepackage{float}
\usepackage{listings}
\usepackage{booktabs}
\usepackage{setspace}%使用间距宏包
\usepackage{mathtools}
\usepackage{amsfonts}

\begin{document} 
\title{homework6}
	\author{11611118 郭思源}  



\begin{spacing}{1.5}%%行间距变为double-space

\section{Question 1}

Prove that a discrete state space stochastic process which satisfies Markovian property has
\[
P^{(n+m)} = P^{(n)}P^{(m)}
\]
and thus
\[
P^{(n)} = P^{(n-1)}P = P\times\cdots\times P = P^n,
\]
where $P^{(n)}$ denotes the matrix of $n$-step transition probabilities and $P^n$ denotes the $n$th power of the matrix $P$.
$\\\\$
\subsection{The probability : }
Let the $n$-step transition probabilities be denoted by 
$p_{ij}^{(n)}$: the probability that a process in state $i$ will be in state $j$ after $n$ additional transitions. That is,
\[
p_{ij}^{(n)} 
= 
P\{X_{n+m}=j|X_m=i\} \ n \geq 0,\ i,\ j \geq 0, 
\]
This probability does not depend on $m$, either!
\[p_{ij}^{(1)} = p_{ij}\]

\newpage
\subsection{Proof : }
\begin{equation*}
\begin{aligned}
p_{ij}^{(n+m)} 
	&= P\{X_{n+m}=j|X_0=i\} \\
	&= \displaystyle \sum_{k=0}^{\infty} 
		P\{X_{n+m}=j,X_{n}=k|X_0=i\} \\
	&= \displaystyle \sum_{k=0}^{\infty} 
		P\{X_{n+m}=j|X_{n}=k,X_0=i\} P\{X_{n}=k|X_{0}=i\} \\
	&= \displaystyle \sum_{k=0}^{\infty} 
		P\{X_{n+m}=j|X_{n}=k\} P\{X_{n}=k|X_{0}=i\} \\
	&= \displaystyle \sum_{k=0}^{\infty}
		p_{ik}^{(n)}p_{kj}^{(m)} .
\end{aligned}
\end{equation*}
$\\$In matrix form, we have
\[P^{(n+m)} = P^{(n)}P^{(m)}\]
and thus
\[P^{(n)} = P^{(n-1)}P^{}=P\times \dots \times P=P^n,\]
where $P^{(n)}$ denotes the matrix of $n$-step transition probabilities and $P^{(n)}$ denote the $n$th power of the matrix $P$.

\begin{center}
$p_{ij}^{(n+m)} 
= \displaystyle \sum_{k=0}^{\infty}p_{ik}^{(n)}p_{kj}^{(m)} 
\quad $ for all $ n,m \geq 0 , \ $all $ i,j \geq 0.$
\end{center}
In that, we have assumed that, without loss any generality, the state space $E$ is \{0, 1, 2, · · · \}. If we do not assume this, and let the state space be $E$, then the equation should be written as
\begin{center}
$p_{ij}^{(n+m)} 
= \displaystyle \sum_{k\in E}^{}p_{ik}^{(n)}p_{kj}^{(m)} 
\quad $ for all $ n,m \geq 0 , \ $all $ i,j \in E.$
\end{center}


\newpage
\section{Question 2}

Consider adding a pizza delivery service as an alternative to the dining halls. \\Table~\ref{T:hw-6-1} gives the transition percentages based on a student survey. Determine the long-term percentages eating at each place. Try several different starting values. Is equilibrium achieved in each case? If so, what is the final distribution of students in each case?

\begin{table}[htb]
\centering
    \begin{tabular}{ccccc}  
    &                                         
    &  \multicolumn{1}{c}{}                                                 
    &  \multicolumn{1}{c}{Next state}                                    
    &  \multicolumn{1}{c}{}                                                  

    \\ \cline{2-5} \multicolumn{1}{c|}{}        
    &  \multicolumn{1}{c|}{}                   
    &  \multicolumn{1}{c}
	   {\begin{tabular}[c]{@{}c@{}}Grease\\Dining Hall\end{tabular}} 
	&  \multicolumn{1}{c}
	   {\begin{tabular}[c]{@{}c@{}}Sweet\\ Dining Hall\end{tabular}} 
	&  \multicolumn{1}{c|}
	   {\begin{tabular}[c]{@{}c@{}}Pizza\\ delivery\end{tabular}} 
	
	\\ \cline{2-5} \multicolumn{1}{c|}{}              
    &  \multicolumn{1}{c|}{Grease Dining Hall} 
    &  0.25 
    &  0.25              
    &  \multicolumn{1}{c|}{0.50}
    
    \\ \multicolumn{1}{c|}{Present state} 
    &  \multicolumn{1}{c|}{Sweet Dining Hall}  
    &  0.10
    &  0.30         
    &  \multicolumn{1}{c|}{0.60}                                           
    
    \\ \multicolumn{1}{c|}{}              
    &  \multicolumn{1}{c|}{Pizza delivery}    
    &  0.05                                 
    &  0.15                                   
    &  \multicolumn{1}{c|}{0.80}
    
    \\ \cline{2-5}  
   	\end{tabular}
\caption{Survey of dining at College USA}\label{T:hw-6-1}
\end{table}

\begin{center}
\[ {p}= 
\begin{bmatrix} 
0.25 & 0.25 & 0.50
\\[8pt] 
0.10 & 0.30 & 0.60
\\[8pt] 
0.05 & 0.15 & 0.80
\end{bmatrix} \] 
\end{center}
from $qp = q$, we have :
\begin{center}
\[ {q}= 
\begin{bmatrix} 
0.0741 & 0.1852 & 0.7407 
\end{bmatrix} \] 
\end{center}
So that we can see for every case, assume that the total number of customers are $n$, then in the equilibrium, Grease Dining Hall has $0.0741n$ customers, Sweet Dining Hall has $0.1852n$ customers, Pizza delivery has $0.7407n$ customers.

\end{spacing}

\end{document}